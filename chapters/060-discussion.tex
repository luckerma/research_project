\chapter{Discussion}\label{chap:discussion}
This chapter provides a critical analysis of the methods, challenges, and future prospects in the field of MTMCT. It compares the mentioned approaches, discusses the gaps and limitations of these methodologies, and mentions the ethical implications and future advances that could revolutionize the field of MTMCT.

\section{Summary of Methods}\label{sec:summary_of_methods}
Early research in MTMCT relied primarily on Bayesian classification and simple network models. These basic methods were instrumental in kick-starting research in the field, but they were limited by robustness and error susceptibility, especially in challenging scenarios such as occlusions or varied poses.

The advent of CNNs marked a significant shift, introducing deep learning to object detection and significantly improving performance. Building on CNN, R-CNN and its successors, including Fast R-CNN and Faster R-CNN, improved efficiency and reduced computational overhead. They introduced more effective RoI handling.

The introduction of real-time object detection frameworks such as YOLO and SSD was a game changer. YOLO, with its single forward pass image processing, revolutionized real-time object detection, and the multiple feature maps of SSD effectively addressed the challenge of varying object sizes.

Data association techniques such as the Hungarian Algorithm, JPDAF, and POM were critical to maintaining object ID over time and across different camera views.

Various tracking paradigms such as TbD, TbR, TbS, and TbA have emerged, each addressing specific aspects of tracking with their unique advantages. Furthermore, single-shot approaches such as SMOT and JDE have streamlined the process by integrating intra-camera detection and tracking in a single-step, emphasizing speed and simplicity toward real-time tracking. In addition, frameworks such as FairMOT, FCDSC, and JDE have set milestones for real-time tracking applications.

The integration of graph-based approaches and neural networks marked another leap forward, providing robust frameworks for data association, especially beneficial for long-term matching and in challenging occlusion scenarios. Neural networks introduced an end-to-end approach, eliminating the need for hand-crafted features and enabling more efficient data association.

The latest transformer-based models brought significant improvements in accuracy and efficiency. These models excel at handling multiple objects across frames.

\section{Gaps and Limitations}\label{sec:gaps_and_limitations}
While each of the aforementioned approaches has unique advantages, they also have inherent limitations. The following sections discuss the gaps and limitations of these methods.

Feature extraction and data association techniques such as SIFT, HOG, the Hungarian Algorithm, JPDAF and POM have been instrumental in the development of MTMCT systems. However, these methods are not able to handle scenarios that increase in complexity with occlusions and varying poses. Furthermore, these methods are computationally expensive, which is a significant limitation in real-time applications.

The performance of the Kalman Filter degrades significantly in the presence of abrupt motion changes or maneuvering targets. On the other hand, MHT, known for its robustness to multiple targets and false alarms, faces computational challenges. As the number of targets and hypotheses increases, the increasing computational complexity of MHT makes it less practical for real-time applications in dense environments.

Existing datasets and challenges still focus primarily on intra-camera tracking. In addition, there are no challenges that focus solely on real-time tracking within a multi-camera system. These challenges would be beneficial for the advancement of research in these areas, as developers would be motivated to develop new methods and algorithms to compete in these challenges.

Intensive research on intra-camera detection and tracking frameworks such as YOLO, Faster R-CNN, SSD and DeepSORT has led to significant progress in these areas. However, these methods are not optimized for inter-camera tracking, which requires at least one step further or a completely different approach. The lack of a unified framework for inter-camera tracking is a significant gap in current research.

Single-shot approaches such as SMOT and JDE have been instrumental in simplifying the tracking process by integrating detection and tracking into a single-step, increasing speed and efficiency. However, the prioritization of speed compromises accuracy, especially when detecting small or overlapping objects.

Graph-based approaches such as DyGLIP and ReST are powerful frameworks for data association, especially in occlusion scenarios. However, graph-based approaches are computationally intensive, which is a significant limitation in real-time applications. With the introduction of FCDSC, which considers only a sub-graph at a time, the computational complexity of graph-based approaches has been significantly reduced, finally making them feasible for real-time applications.

Attention models and transformers are still in the early stages of development and have not been extensively explored in MTMCT. While these models have shown promising results in other domains, their potential in MTMCT has not yet been fully realized and in particular, their use in real-time applications is not yet feasible.

While a lot of progress has been made in the area of short-term trackers, long-term trackers have been overlooked, even though they are closer to real-world scenarios. Therefore, \textcite{Zadeh21} call for the development of trackers that are capable of re-ID targets over a long period of time. In addition, the authors state that generic visual trackers are needed to quickly adapt to unseen targets in real-world scenarios.

Furthermore, most approaches lack the ability to handle objects of different classes and both non-overlapping and overlapping camera views. The ability to handle objects of different classes is especially important in scenarios where both people and vehicles are present. The ability to handle non-overlapping and overlapping camera views is important in scenarios where the camera setup is not known in advance and the cameras are not calibrated. Thus, there is no \textit{one-size-fits-all} approach yet. The development of such a unified framework could also be encouraged by the introduction of a challenge focusing on these aspects.

\section{Future Research}\label{sec:future_research}
The advancement of online and real-time methods is critical given the increasing demand for instantaneous and accurate tracking in various real-time applications. To make significant progress in this area, several research directions need to be explored.

Algorithms should ideally balance speed and accuracy, providing accurate tracking information quickly. Emphasis on lightweight neural network architectures could lead to models that maintain high accuracy while reducing computational complexity, which is critical for real-time applications. In addition, the integration of MTMCT systems with edge computing offers a promising way to improve real-time processing~\cite{Yu17}. By processing data closer to its source, latency can be significantly reduced. Optimizing MTMCT algorithms for edge devices, which often have limited computational resources, would ensure efficient operation and faster data processing.

\textcite{Amosa23} mention other aspects that need to be addressed in the future. For example, they call for a unified evaluation metric for multi-camera systems. Currently, evaluation is still based on single-camera tracking metrics, which are outdated for evaluating a multi-camera system. They also suggest investigating semi- or unsupervised learning approaches in the context of MTMCT, which would reduce the need for labeled data. They also consider the integration of language-vision models, which could improve tracking accuracy by incorporating textual information into the tracking process.

Effective resource management also plays a critical role in real-time MTMCT systems. Developing algorithms to dynamically allocate computational resources based on the complexity of the tracking scene would ensure optimal use of available processing power. Effectively dealing with varying data quality, crowd density, and environmental conditions in real-time is another challenge that needs to be addressed. Algorithms that can adapt to these variations in real-time, while maintaining accuracy across different scenarios, would significantly improve the robustness of real-time tracking systems.

Low-latency communication protocols are essential, especially for systems that require real-time data synchronization and analysis from multiple cameras. Research in this area could take advantage of the potential of advanced technologies, such as 5G, for high-speed data transmission, which is essential for synchronizing and analyzing data from multiple sources in real-time.

Finally, as privacy concerns grow, it is increasingly important to develop real-time tracking systems that respect individual privacy. Techniques such as on-device processing, anonymization of tracking data such as face blurring, and secure transmission methods could be key areas of research to ensure privacy-preserving real-time tracking.

In summary, improving online and real-time capabilities in MTMCT requires a multifaceted approach that includes algorithmic innovation, hardware optimization, and balancing the demands of speed, accuracy, and privacy. Addressing these research areas will lead to more responsive, efficient, and reliable real-time tracking solutions that meet the dynamic needs of modern applications.

\section{Ethical and Privacy Concerns}\label{sec:ethical_and_privacy_concerns}
Future advances should balance technological progress with ethical considerations, ensuring that privacy and ethical standards are protected. Research in the areas of synthetically generated datasets and edge computing could potentially address privacy concerns. To name just two important examples of these concerns: The developer of the YOLO framework stopped work on the project due to ethical dilemmas, fearing that his work could be used for military applications~\cite{Synced20}. Similarly, the DukeMTMC dataset was withdrawn due to privacy concerns~\cite{Harvey21}. These cases highlight the complex interplay between technological progress and ethical responsibility.