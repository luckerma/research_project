\chapter{Discussion}\label{chap:discussion}
This chapter delves into a critical analysis of the methods, challenges, and future prospects in the field of MTMCT. It compares the mentioned approaches, discusses gaps and limitations of these methodologies, and mentions the ethical implications and future advancements that could revolutionize the field of MTMCT.

\section{Summary of Methods}\label{sec:summary_of_methods}
Early research in MTMCT, primarily utilized Bayesian classification and network models. These foundational methods were instrumental in kickstart research in the field, though they were constrained by limitations in robustness and susceptibility to errors, particularly in challenging scenarios like occlusions or varied poses.

The advent of CNNs marked a significant shift, introducing deep learning to object detection and substantially elevating performance. Building upon CNNs, R-CNN and its successors, including Fast R-CNN and Faster R-CNN, enhanced efficiency and reduced computational overhead. They introduced more effective region-of-interest handling.

The introduction of real-time object detection frameworks like YOLO and SSD was a game-changer. YOLO, with its single forward pass image processing, revolutionized real-time detection, and the multiple feature maps of SSD effectively addressed the challenge of varying object sizes.

Data association techniques such as the Hungarian Algorithm, JPDAF, and POM have been pivotal in maintaining object identity over time and across different camera views.

Various tracking paradigms emerged such as tracking-by-detection, tracking-by-regression, tracking-by-segmentation, and tracking-by-attention each addressed specific aspects of tracking with their unique advantages. Furthermore, single-shot approaches like SMOT and JDE streamlined the process by integrating intra-camera detection and tracking in a single step, emphasizing speed and simplicity.

The integration of graph-based approaches and neural networks marked another leap forward, offering robust frameworks for data association, especially beneficial for long-term matching and in challenging scenarios with occlusions. Neural networks introduced an end-to-end approach, eliminating the need for hand-crafted features and enabling more efficient data association.

The most recent transformer-based models brought significant improvements in accuracy and efficiency. These models excel in handling multiple objects across frames.

\section{Gaps and Limitations}\label{sec:gaps_and_limitations}
While every mentioned approach has its unique advantages, they also have inherent limitations. The following sections discuss the gaps and limitations of these methodologies.

Feature extraction and data association techniques like SIFT, HOG, the Hungarian algorithm, JPDAF and POM are instrumental in the development of MTMCT systems. However, these methods are not able to handle scenarios which constantly growing in complexity with occlusions and varying poses. Furthermore, these methods are computationally expensive, which is a significant limitation in real-time applications.

The performance of the Kalman Filter significantly degrades in the presence of abrupt motion changes or maneuvering targets. On the other hand, MHT, known for its robustness in handling multiple targets and false alarms, faces computational challenges. As the number of targets and hypotheses increases, the increasing computational complexity of MHT makes it less practical for real-time applications in dense environments.

The existing datasets and challenges still mainly focus on intra-camera tracking and there is no challenge yet that focuses on inter-camera tracking within multiple-camera systems. Furthermore, a challenge that focuses solely on real-time tracking is also missing. Both of these challenges would be beneficial to advance the research in these areas, due to developers would be motivated to develop new methods and algorithms to compete in these challenges.

The intense research on intra-camera detection and tracking frameworks like YOLO, Faster R-CNN, SSD and DeepSORT has led to significant advancements in these areas. However, these methods are not optimized for inter-camera tracking, that needs at least one step further or a completely different approach. The lack of a unified framework for inter-camera tracking is a significant gap in the current research.

Single shot approaches like SMOT and JDE have been instrumental in simplifying the tracking process by integrating detection and tracking in a single step and boosting speed and efficiency. However, these methods lack the robustness and accuracy of two-stage approaches, which is a significant limitation in challenging scenarios.

Graph based approaches like DyGLIP and ReST are powerful frameworks for data association, especially in scenarios with occlusions. However graph-based approaches are computationally intensive that is a significant limitation in real-time applications. With the introduction of FCDSC that only considers a sub-graph each step, the computational complexity of graph-based approaches has been reduced significantly and finally made them feasible for real-time applications.

Attention models and transformers are still in the early stages of development and have not been extensively explored in MTMCT. While these models have shown promising results in other domains, their potential in MTMCT is yet to be fully realized and especially their deployment in real-time applications is still not feasible yet.

Furthermore most approaches lack in the ability of handling objects of different classes and both non- and overlapping camera views. The ability of handling objects of different classes is especially important in scenarios where both people and vehicles are present. The ability of handling non- and overlapping camera views is important in scenarios where the camera setup is not known in advance and the cameras are not calibrated. So there is no \textit{one size fits all} approach yet. The development of such unified framework could also be boosted by the introduction of a challenge that focuses on these aspects.

\section{Future Research}\label{sec:future_research}
The advancement of online and real-time methods is critical, considering the increasing demand for instant and accurate tracking in various real-time applications. To achieve significant progress in this area, several research directions need to be explored.

The algorithms should ideally strike a balance between speed and accuracy, providing precise tracking information swiftly. Emphasis on lightweight neural network architectures could lead to models that maintain high accuracy while reducing computational burden, which is crucial for real-time applications. Additionally, integrating MTMCT systems with edge computing offers a promising avenue to enhance real-time processing. By processing data closer to its source, latency can be substantially reduced. Optimizing MTMCT algorithms for edge devices, which often have limited computational resources, would ensure efficient operation and quicker data processing.

Effective resource management also plays a critical role in real-time MTMCT systems. Developing algorithms for dynamic allocation of computational resources, depending on the complexity of the tracking scene, would ensure the optimal use of available processing power. Handling varying data quality, crowd density, and environmental conditions effectively in real-time is another challenge that needs addressing. Algorithms capable of adapting to these variations in real time, while maintaining accuracy across different scenarios, would significantly enhance the robustness of real-time tracking systems.

Low-latency communication protocols are vital, especially for systems where data synchronization and analysis from multiple cameras are required promptly. Research in this domain could leverage the potential of advanced technologies like 5G for high-speed data transmission, essential for synchronizing and analyzing data from multiple sources in real time.

Lastly, with growing concerns around privacy, developing real-time tracking systems that respect individual privacy is increasingly important. Techniques such as on-device processing, anonymization of tracking data for example blurring faces, and secure transmission methods could be key areas of research to ensure privacy-preserving real-time tracking.

In conclusion, enhancing online and real-time capabilities in MTMCT involves a multifaceted approach, encompassing algorithmic innovation, hardware optimization, and balancing the demands of speed, accuracy, and privacy. Addressing these research areas will lead to more responsive, efficient, and reliable real-time tracking solutions, aligning with the dynamic needs of modern applications.

\section{Ethical and Privacy Concerns}\label{sec:ethical_and_privacy_concerns}
Future advancements should balance technological progress with ethical considerations, ensuring privacy and ethical standards are secured. Research in the area of synthetically generated datasets and edge computing could potentially address the privacy concerns. Just to mention two significant examples for the concerns: The developer of the YOLO framework stopped working on the project due to ethical dilemmas fearing that his work could be used for military applications~\cite{Synced20}. Similarly, the DukeMTMC dataset, was withdrawn over privacy issues~\cite{Harvey21}. These cases underscore the complex interplay between technological advancement and ethical responsibility.