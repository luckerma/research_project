\chapter{Introduction}\label{chap:introduction}
Multi-Target Multi-Camera Tracking (MTMCT) is an essential field of research in computer vision, with significant applications ranging from video surveillance and traffic monitoring to sports analysis and crowd management. By simultaneously tracking multiple objects across various camera views, MTMCT systems aim to provide a comprehensive understanding of the scene dynamics and interactions.

The advent of deep learning and other advanced algorithms has revolutionized the field of MTMCT, especially in the last years, enabling faster, more accurate and reliable tracking in complex environments. In particular, online and real-time tracking methods have emerged as a critical area of focus, given their potential to provide timely and actionable insights in various real-world applications.

Even though Single-Target Single-Camera Tracking (ST-SCT) as well as Multi-Target Single-Camera Tracking (MT-SCT) has been extensively studied, MTMCT is still a relatively new and challenging, but also promising area of research. The complexity of MTMCT is significantly higher than ST-SCT and MT-SCT, due to the need to simultaneously track multiple objects across multiple cameras.

Single-Target Multi-Camera (ST-MCT) is a insignificant field of research, because if the use-case requires multiple cameras, it is almost always necessary to track multiple targets. Therefore, this project will not cover the special case of ST-MCT.

This research project aims to provide a comprehensive review of the state-of-the-art in MTMCT, with a special focus on online and real-time tracking methods. Latest trends, technologies, and challenges in this field are explored, drawing insights from recent research papers and studies. This review highlights the significant advancements made in MTMCT and identifies the gaps and opportunities for future research.

The rest of this project is structured as follows. Chapter~\ref{chap:background} provides an overview of the key challenges and issues in MTMCT, along with a discussion of the datasets, metrics, and components of an MTMCT system. Futhermore it explains the basic concepts of object detection and tracking. Chapter~\ref{chap:literature_review} presents a detailed review of the literature on MTMCT, with the main focus on online and real-time tracking methods with static cameras. Chapter~\ref{chap:discussion} compares and contrasts the different methods reviewed in the previous sections, identifies the gaps and limitations in current research, and suggests areas for future research. While considering the ethical and privacy concerns related to MTMCT, it also discusses the need for regulations and guidelines. Finally, Chapter~\ref{chap:conclusion} concludes the project with a summary of the key findings and insights, along with stating the future directions and challenges for research in this area.

\section{Definition of MTMCT}\label{sec:definition_of_mtmct}
MTMCT is an integration of object detection and tracking methodologies to simultaneously track multiple predefined objects of interest across various camera views. The objective of MTMCT is to maintain a coherent understanding of the identities (IDs) of the objects and their paths as they move through the fields of view of different cameras. The objects of interest are often people and vehicles, but in theory can be any moving object. The camera setup differs from one application to another, but typically consists of multiple cameras with either overlapping, non-overlapping, or partially overlapping fields of view. The cameras may be static or moving, and may be placed at different heights and angles. The cameras may also have differing technical specifications like resolution, frame rate, and field of view (FOV).

\section{Importance of MTMCT}\label{sec:importance_of_mtmct}
MTMCT plays a crucial role in various real-world applications. In video surveillance, it is used to monitor and analyze the movement of individuals or vehicles across different cameras, which can be vital for security and forensic analysis. In sports analysis, MTMCT can provide valuable insights by tracking the movement and interaction of players across different camera angles. In traffic monitoring, MTMCT can help manage traffic flow and detect incidents by tracking vehicles as they move through different camera views.

Furthermore, the need for online and real-time tracking in these applications is imperative. Real-time processing of data streams from multiple cameras and providing instantaneous tracking results are essential to make timely and actionable insights, which is particularly relevant in scenarios like accident prevention, control of traffic flow, crime detection, and real-time sports analysis.

\section{Objective of Research Project}\label{sec:objective_of_review}
First, the basics concepts of SCT are explained to provide a foundation for understanding MTMCT. The primary objective of this project is to provide a comprehensive overview of proposed methods and technologies for MTMCT and review the current state-of-the-art in MTMCT, with a special focus on online and real-time tracking methods. Through an extensive literature review, the aim is to explore the latest trends, technologies, and challenges faced in this field, and provide insights drawn from recent research papers and studies. By highlighting the significant advancements made in MTMCT, the intend is to identify the gaps in current research and outline potential avenues for future exploration, while keeping in mind the ethical and privacy concerns related to MTMCT.

\section{Related Work}\label{sec:related_work}
The work of \textcite{Zheng16c} focuses on the past, present and future of person re-identification (re-ID), that is the task of identifying a person across multiple cameras for example if the person leaves and re-enters the field of view of a camera or a person is lost for a short time. The paper covers hand-crafted algorithms as well as deep learning approaches for both image- and video-based re-ID. Furthermore, important datasets are covered, quickly explained and the approaches are evaluated. Although this paper gives a good overview, it was published in \citeyear{Zheng16c} and therefore does not cover the latest research in this field, which will be covered by this project.

Two years later \citetitle{Iguernaissi18}~\cite{Iguernaissi18} was published which gives an overview of multi-camera tracking methods. The review covers the most important methods and dataset in the field of tracking people in a multi-camera system. However, it does not cover the task of tracking vehicles and is limited to approaches that were released until \citeyear{Iguernaissi18}.

One chapter of the doctoral thesis of \textcite[Chapter 5]{Tian19}, published in \citeyear{Tian19}, revolves around the topic of tracking multiple objects and gives a state-of-the-art overview of this field. It does not cover the topic of multi-camera tracking. However, it provides a mathematical insight into the topic of tracking multiple objects in a single-camera system.

The extensive survey \citetitle{Zadeh21}~\cite{Zadeh21} by \citeauthor{Zadeh21} delineates the evolution and the state (\citeyear{Zadeh21}) of deep learning-based visual tracking methods, categorizing these methods based on their network architecture, training processes, and learning procedures. It provides a detailed examination of various deep learning architectures and custom networks, each contributing to the efficiency and robustness of visual trackers. It analyzes the challenges faced by DL-based trackers and the solutions proposed to address them. The survey also offers a comprehensive comparison of well-known single-object visual tracking datasets, evaluating and analyzing state-of-the-art DL-based methods across a range of tracking scenarios. While this survey focus on single-object tracking, their insights into the advancements in deep learning architectures and methodologies provide a valuable context for the study on MTMCT that also adapts and applies mentioned approaches. This survey thus serves as an important source understanding the broader landscape of deep learning applications in visual tracking in general.

The most recent and comprehensive review of MTMCT was published in \citeyear{Amosa23} by \textcite{Amosa23}. It provides a detailed overview of the state-of-the-art in MTMCT, covering the latest trends, technologies, and challenges in this field. However, the mentioned review gives a broader overview and does not focus on online and real-time tracking methods, which is a main aspect of this project. Futhermore, this research project aims to provide an easier introduction to the field of MTMCT by first explaining the basics before diving into the details of the latest research.

The integration of edge computing in IoT, as explored in \citetitle{Yu17}~\cite{Yu17}, provides valuable insights for the MTMCT domain. This survey highlights how edge computing significantly reduces latency and balances network traffic, which are critical for real-time data processing in IoT networks. Such capabilities are directly relevant to the challenges faced in MTMCT, especially when dealing with high volumes of data from multiple cameras. The discussion of the paper on distributed computational nodes and their role in supporting real-time analysis and decision-making offers a parallel to the computational needs in MTMCT. Incorporating edge computing concepts could potentially enhance the performance and efficiency of MTMCT systems, particularly in scenarios requiring quick processing of complex data from various sources.