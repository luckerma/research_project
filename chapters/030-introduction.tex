\chapter{Introduction}\label{chap:introduction}
Multi-Target Multi-Camera Tracking (MTMCT) is a crucial research area in computer vision with important applications in video surveillance, traffic monitoring, sports analysis and crowd management. By simultaneously tracking multiple objects across multiple camera views, MTMCT systems aim to provide a comprehensive understanding of scene dynamics and interactions.

The advent of deep learning and other advanced algorithms has revolutionized the field of MTMCT, especially in recent years, enabling faster, more accurate and reliable tracking in complex environments. In particular, online and real-time tracking methods have emerged as a critical area of focus due to their potential to provide timely and actionable insights in various real-world applications.

Even though Single-Target Single-Camera Tracking (STSCT) as well as Multi-Target Single-Camera Tracking (MTSCT) have been extensively studied, MTMCT is still a relatively new and challenging, but promising area of research. The complexity of MTMCT is significantly higher than that of STSCT and MTSCT due to the need to track multiple objects simultaneously using multiple cameras.

Single-Target Multi-Camera is an insignificant area of research because if the use case requires multiple cameras, it is almost always necessary to track multiple targets. Therefore, this project will not cover this particular case.

This research project aims to provide a comprehensive review of the milestones and state-of-the-art in MTMCT, with a special focus on online and real-time tracking methods. The latest trends, technologies and challenges in the field will be explored, drawing insights from recent research papers and studies. It also highlights the significant advances that have been made in MTMCT and identifies the gaps and opportunities for future research.

The rest of this project is structured as follows: The following sections of this chapter define MTMCT and its importance, as well as the objective of this research project and related work. Chapter~\ref{chap:background} provides a brief overview of the basics of MTMCT to provide a foundation for understanding the rest of this project. Chapter~\ref{chap:literature_review} mentions the previous milestones and reviews the current state-of-the-art in MTMCT, with a special focus on online and real-time tracking methods. Chapter~\ref{chap:discussion} provides a critical analysis of the methods, challenges, and future prospects in the field of MTMCT. Chapter~\ref{chap:conclusion} concludes by summarizing the main findings and outlining potential avenues for future research.

\section{Definition of MTMCT}\label{sec:definition_of_mtmct}
MTMCT is an integration of object detection and tracking methods to simultaneously track multiple objects of interest across different camera views. The goal of MTMCT is to maintain a coherent understanding of the identities (IDs) of the objects and their paths as they move through the fields of view (FOV) of different cameras. The objects of interest are often people and vehicles, but could theoretically be any moving object. The camera setup varies from application to application, but typically consists of multiple cameras with either overlapping, non-overlapping, or partially overlapping FOVs. The cameras can be static or moving, and can be placed at different heights and angles. The cameras can also be heterogeneous, with different technical specifications such as resolution, frame rate, and FOV.

\section{Importance of MTMCT}\label{sec:importance_of_mtmct}
MTMCT plays a critical role in several real-world applications. In video surveillance, it is used to monitor and analyze the movement of people or vehicles across multiple cameras, which can be critical for security and forensic analysis. In sports analysis, MTMCT can provide valuable insights by tracking the movement and interaction of players across multiple camera angles. In traffic monitoring, MTMCT can help manage traffic flow and detect incidents by tracking vehicles as they move through different camera views.

In addition, the need for online and real-time tracking is imperative in these applications. Real-time processing of data streams from multiple cameras and the provision of instant tracking results are essential to provide timely and actionable insights, which is particularly relevant in scenarios such as accident prevention, traffic flow control, crime detection, and real-time sports analysis.

\section{Objective of Research Project}\label{sec:objective_of_review}
First, the basic concepts of object detection and tracking are explained to provide a foundation for understanding MTMCT. The primary objective of this project is to provide a comprehensive overview of proposed methods and technologies for MTMCT and to review the current state-of-the-art in MTMCT, with a special focus on online and real-time tracking methods. Through an extensive literature review, the goal is to explore the previous milestones, recent trends, technologies, and challenges in the field and provide insights from recent research papers and studies. By highlighting the significant advances made in MTMCT, it is intended to identify the gaps in current research and outline potential avenues for future exploration, while keeping in mind the ethical and privacy concerns associated with MTMCT.

\section{Related Work}\label{sec:related_work}
The work of \textcite{Zheng16c} focuses on the past, present and future of person re-identification (re-ID), which is the task of identifying a person across multiple cameras, for example when a person leaves and re-enters the FOV of a camera, or when a person is briefly lost by the detection framework. The paper covers hand-crafted algorithms as well as deep learning approaches for both image- and, more importantly, video-based re-ID. It also quickly explains important datasets and evaluates the approaches. Although this paper gives a good overview, it was published in \citeyear{Zheng16c} and therefore does not cover the latest research in the field, which will be covered by this project.

Two years later, \citetitle{Iguernaissi18}~\cite{Iguernaissi18} was published, which gives an overview of multi-camera tracking methods. The review covers the most important methods and datasets in the field of tracking people in a multi-camera system. However, it does not cover the task of tracking vehicles and is limited to approaches that were released until \citeyear{Iguernaissi18}.

A chapter of the doctoral thesis of \textcite[Chapter 5]{Tian19}, published in \citeyear{Tian19}, revolves around the topic of tracking multiple objects and gives a state-of-the-art overview of the field. It does not cover the topic of multi-camera tracking, but provides a mathematical insight into the topic of tracking multiple objects in single-camera systems.

The survey \citetitle{Zadeh21}~\cite{Zadeh21} conducted by \citeauthor{Zadeh21} in \citeyear{Zadeh21} describes the evolution and state of deep learning-based visual tracking methods, categorizing these methods based on their network architecture, training processes, and learning procedures. It provides a detailed examination of various deep learning architectures and custom networks, each of which contributes to the efficiency and robustness of visual trackers. It analyzes the challenges faced by deep learning-based trackers and the solutions proposed to address them. The survey also provides a comprehensive comparison of well-known single-object visual tracking datasets, evaluating and analyzing state-of-the-art deep learning-based methods across a range of tracking scenarios. While this survey focuses on single-object tracking, its insights into advances in deep learning architectures and methods provide a valuable context for the study of MTMCT, which also adapts and applies the aforementioned approaches. Thus, this survey serves as an important resource for understanding the broader landscape of deep learning applications in visual tracking in general.

The most recent and comprehensive review of MTMCT was published in \citeyear{Amosa23} by \textcite{Amosa23}. It provides a detailed overview of the state-of-the-art in MTMCT, covering the latest trends, technologies, and challenges in the field. However, the mentioned review gives a broader overview and does not focus on online and real-time tracking methods, which is a main aspect of this project. Furthermore, this research project aims to provide an easy introduction to the field of MTMCT by first explaining the basics before diving into the details of the latest research.

The integration of edge computing into the IoT, as explored in \citetitle{Yu17}~\cite{Yu17}, provides valuable insights into the MTMCT domain. This survey highlights how edge computing significantly reduces latency and balances network traffic, which are critical for real-time data processing in IoT networks. Such capabilities are directly relevant to the challenges faced in MTMCT, especially when dealing with large amounts of data from multiple cameras, it provides a parallel to the computational needs in MTMCT.