\chapter{Structure}

\section{Citations}

\subsection{General}

\cite{Amosa23}: Current Trends in MCMOT. State of the Art. A lot of basic and advanced knowledge. Good for introduction. Analyzes 30 MCT algorithms.

\cite{Tian19}: General description of multi-camera tracking. State of the Art, Markov Process, graph partition theory, tracking by joint constraints.

\cite{Khan01}: Tracking people in multiple uncalibrated cameras. Discover spatial relationships between the camera FOVs. Tested on PETS 2001.

\subsection{Beginning}

\cite{Cai99}: First approaches of tracking humans in multi camera network. Already done in 1999 with real-time tracking. Automatic camera switching. Bayesian classification schema.

\cite{Chang01}: Bayesian modality fusion to track multiple people in an indoor environment. Tries to fix already known occlusion problem.

\subsection{Real-time}

\cite{Ren17}: Faster R-CNN. Towards Real-Time Object Detection. Region Proposal Network (RPN). RPN is trained end-to-end. Attention mechanism. 5-17 fps on GPU. Two modules (first region proposal, second detector). Sharing convolutional features.

\cite{Wang20a}: Toward Real-Time. Only multi-object tracking. Introduces JDE (Joint learning of detection and embedding). Very important paper (first real-time MOT system). Single-shot detector

\cite{Wang21}: Indoor scene, multiple top-view \textbf{fisheye} cameras. Possible to cover large space, less occlusion among objects. People detection and tracking. Calibrate cameras, real time (FPS of about 10) without GPU support.

\cite{Chen22}: Real-time distributed MCMOT system. City-scale scenario. Keeping communication and computing costs of each device low. Installs smart stations on the roadside and connects them to maintain communication. Decentralized Tracking. Kalman filter and hungarian algorithm. YoloX and DeepSORT.

\cite{Zhang21}: FairMOT, one-shot tracker (anchor-free style). Tackles issue of object detection against re-ID. Re-ID often threated as secondary task. Reasons behind failure: anchors, feature sharing, feature dimension.

\cite{Tesfaye19}: Multiple non-overlapping cameras using fast-constrained dominant set clustering (FCDSC). Three-layer hierarchical approach. Orders of magnitudes faster than existing methods. Can be used in conjunction with re-id algorithms. Good graphics in paper.

\subsection{VOT}

\cite{Kristan21}: VOT21 Challenge Results. Considers single-camera, single-target, model-free tracking. VOT-RT2021 focuses on real-time RGB tracking. Requires predicting bounding boxes. Top two trackers: TrasT\_M and STARK\_RT.

\cite{Kristan22}: VOT22 Challenge Results. Considers single.camera, single-target. VOT-RT2022 focuses on real-time RGB tracking, VOT-RTs by segmentation, VOT-RTb by bounding boxes. Goes beyond previous challenges (updating datasets). Real-time tracking at 20fps. Top trackers: MS\_AOT and OSTrackSTB.

\cite{Kristan23} VOTS23 Challenge Results. First year considering multiple-target tracking challenge. Explores short- and long-term at once. Only one challenge for all. Does not distinguish between these scenarios. Success is measured in IoU, tracking Quality \\mathbf{Q}, Accuracy, Robustness, NRE, DRE, ADQ. Dataset with challenging situations, wide range and diverse set of objects, object which are a part of other objects.  Also longer videos. 77 trackers submitted, 47 valid. Most trackers applied uniform dynamic model, utilized transformers, general segmentation network SAM. Top tracker: DMAOT built upon VOT22 winner AOT. Best segmentation-based trackers outperformed all bound.box trackers.

\subsection{Dynamic Cameras}

\cite{Zou19}: Tracking multiple vehicles in the front view of an onboard monocular camera. Siamese network with a spatial pyramid pooling. Markov decision process. Effective for real-time long-term tracking. Hungarian algorithm, reinforcement learning.

\cite{Nguyen22b}: Single-Stage Global Association Approach. Dynamic MCMOT (moving cameras in vehicle). Solves fragment-tracking issues. Not relevant for static MCMOT.

\subsection{Person Tracking}

\cite{Chen17b}: Integrating social grouping behavior for tracking pedestrians. Online learned conditional random field (CRF). Non-overlapping cameras.

\cite{Huang23}: Non-overlapping cameras. Pedestrian Tracking. Fix ID-switching issues with long-term feature extraction. OC-SORT + feature extraction.

\cite{Komorowski22}: Soccer Players. Raw detection heat maps. Google Research Football Environment. Multi camera, multi targets. Cameras have fixed positions. Do not use bounding boxes, instead raw input with heat maps. Graph Neural Network. No visual cues, such as jersey numbers. Player movement trajectories and interaction between neighborhood players.

\cite{You21}: Optical-based Pose Association (OPA). Online data association algorithm. Solve the occlusion problem. Take also human pose (see \cite{Li19}) and optical flow into account, not only visual and spatial information. OpenPose, Object Keypoint Similarity, PWC-Net, Kunh-Munkras algorithm.

\subsection{Vehicle Tracking (AI City)}

\cite{Specker22}: Multi-camera vehicle tracking. No real-time tracking. Improve single-camera tracklets. 4th place in 2022 AI City Challenge. Track refinement module. Yolov5 pre-trained on COCO. Using GAN to generate synthetic data. Background filtering. Hierarchical clustering, zones, two rounds of clustering (tracklets separately each possible transition between cameras, akk tracks fro adjacent cameras).

\cite{Lui21}: Inspired \cite{Specker22}. First place in 2021 AI City Challenge. Yolov5 pre-trained on COCO. Most important: Introduces two step clustering (inter-zone, inter-camera clustering).

\cite{Specker21}: Fourth place in 2021 AI City Challenge (Track 3). Occlusion-aware tracking system. Inspired by Stadler.

\cite{Li22a}: Second place in 2022 AI City Challenge (Track 1). No new innovations made on first glance.

\cite{Qian20}: First place in 2020 AI City Challenge (Track 3). Electricity. Efficient vehicle tracking system. Aggregation loss and fast multi-target cross-camera tracking strategy. Weighted inter-class non-maximum suppression.

\cite{Hsu22}: Graph Auto-Encoder and Self-Supervised Camera Link Model. First implementation of GAE in MTMCT. Very interesting paper. Network topology is learned automatically.

\subsection{Re-ID, Data Association and Tracklet Matching}

\cite{Peng16}: Unsupervised cross-dataset transfer learning for person re-id. Unsupervised multi-task dictionary learning (UMDL) model. Uses latent attributes. Asymmetric multi-task learning approach.

\cite{Zhang17}: First time use of hierarchical clustering for person re-id. No online method (needs neighboring frames).

\cite{Lee18}: Online-learning-based person re-id. Fully unsupervised learning method. Systematically  builds camera link model. Two-way GMM fitting. Multi-kernel adaptive segmentation. Multi-shot framework.

\cite{Jiang18}: Orientation-driven person re-id (ODPR). Leverages the orientation cuest and stable torso features to learn a discriminative representation. Also estimates camera topology.Entry/Exit zones are clustered with GMM.

\cite{Hou19}: Locality aware appearance metric (LAAM). Intra- and inter-camera metric for re-ID. Can be applied on top of globally learned re-ID features. Improves tracking accuracy.

\cite{Li19}: State-aware Re-ID. Human pose information is adopted to infer the target state including occlusion status and orientation. State-of-the-art result on Duke-MTMCT.

\cite{Li22b}: Proposes Mutual Information Temporal Weight Aggregated Person Re-ID Model (MI-TWA). Person re-identification. New algorithm. Not so interesting.

\cite{Quach21}: Dynamic Graph Model with Link Prediction. Tackles problem of data association with a dynamic graph model. Better feature representations and able to recover from lost tracks during camera transitions. Works for person and vehicle tracking for overlapping and non-overlapping cameras. First time link prediction and dynamic graph are used together for MCMOT. Attention models.

\cite{Hsu21}: Metadata-Aided Re-ID. Uses metadata information (car type, brand and color) for re-ID. Traffic-aware single-camera tracking. trajectory-based camera link model. Not so interesting.

\cite{He20a}: Tracklet-to-Target Assignment. Solves cross-camera tracklet matching problem by TRACTA. Proposes the Restricted Non-negative Matrix Factorization (RNMF) algorithm. Estimates the number of targets in the whole network. Important paper.

\subsection{Datasets}

\cite{Ristani16}: Largest annotated calibrated data set for MTMC (DukeMTMC).

\cite{Koehl20}: Created MTMCT dataset in GTA V. No privacy issues. 6 cameras over 100 minutes per camera. Largest synthetic dataset for multi camera multi person tracking.

\subsection{Misc}

\cite{Zhang15a}: Tracking framework for multiple interacting targets both overlapping and non-overlapping cameras, raw target trajectory with group state. SVMS, homography-based voting schema, networkflow problem, K-shortest paths algorithm.

\cite{Choi16}: Non-overlapping multiple cameras tracking based on similarity function. Data association method. Similarity based on color appearance and camera topology. Use superpixels for extracting color features generated by Simple Linear Iterative Clustering K-means camera topology learning.

\cite{Yoon18}: Multiple hypothesis tracking (MHT) for multi-camera tracking. Track hypothesis trees. Disjoint views. Status: tracking, searching, end-of-track. Real-time online method (15 fps). Also uses pose of person.

\cite{Nguyen22a}: Mathematical multi-camera tracking approach. Pre-clustering obtained from 3D geometry projections.

\cite{Cheng23}: Utilizes information regarding spatial and temporal consistency. Reconfigurable graph model. Two step approach: Associate all objects across cameras spatially then reconfig into a temporal graph model. Matching object across different views.

\cite{Chen17a}: Equalized Global Graph Model-Based Approach. Improved similarity metric for single- and multiple-camera tracking. SCT and ICT in one step.

\cite{Cho19}: Joint person re-id and camera network topology inference. First framework which jointly solves both problems. Minimal prior knowledge about environment. Multi-shot method implemented as random-forest.

\cite{Chu19}: Joint learning of feature, affinity and multi-dimensional assignment (FAMNet). Online MOT. One deep-network for all three tasks. End-to-end learning.

\section{Approaches}

Single vs Multi Camera Tracking

Static vs Dynamic MCMOT

Single-Stage vs Multi-Stage Tracking

Intra camera vs Inter camera tracking

Local and Global tracklets

Cross-camera tracklet matching problem

Graph Neural Networks, Self-Attention, Transformers

Hierarchical Clustering

Gaussian Mixture Models (GMM)

Tracking by detections (Multi-shot) vs One-shot (Single-shot)

Challenges: Occlusion, perspective changes, changes in lighting, changes in appearances, unknown number of targets in the whole network, unknown number of cameras in which a certain target appears.

Common Pipeline:
\begin{itemize}
    \item Detection
    \item Feature Extraction
    \item Single Camera Tracking
    \item Cross Camera Association
    \item Multi Camera Tracking
\end{itemize}

\section{Die Beschics}

Single Object Detection (SOD)

Multi Object Detection (MOD)

Object Re-Identification (ReID)

Single Camera Tracking (SCT)

Multi Camera Tracking (MCT)

Camera Link Model (CLM)

Trajectories and Tracklets

Fisheye vs Normal Cameras

Online vs Offline Tracking (Online: real-time and frame-by-frame, Offline: post-processing)

Local neighborhood:
Single-camera tracking: Consecutive frames.
Multi-camera tracking: Neighboring cameras.

Different cameras have different technical characteristics.

Appearance features vs Motion features

Datasets:
\begin{itemize}
    \item DukeMTMC
    \item MOTChallenge
    \item AI City Challenge
    \item PETS
    \item CityFlow
\end{itemize}

\section{Composition}

\begin{itemize}
    \item Introduction
    \item Motivation
    \item Technical Background
    \item Problem Statement
    \item State of the Art
    \item Approaches
    \item Challenges
    \item Papers
    \item Further Research
    \item Conclusion
\end{itemize}

\section{Research}

\section{Mentioned Papers}

mentioned in \cite{Specker21}:

D. Stadler and J. Beyerer. Improving multiple pedestrian tracking by track management and occlusion handling. In IEEE Conf. Comput. Vis. Pattern Recog., 2021.

mentioned in \cite{Yoon18}:

Ristani, E., Tomasi, C.: Tracking multiple people online and in real time. Proc. Asian Conf. Computer Vision, Singapore, 2014, pp. 444-459

Wei, S.-E., Ramakrishna, V., Kanade, T., et al.: Convolutional pose machines. Proc. IEEE Conf. Computer Vision and Pattern Recognition, Las Vegas, USA, 2016, pp. 4724-4732

mentioned in \cite{Zhang17}:

Kuhn, H. W. 2010. The hungarian method for the as- signment problem. In 50 Years of Integer Programming.

Zhang, X.; Luo, H.; Fan, X.; Xiang, W.; Sun, Y.; Xiao, Q.; Jiang, W.; Zhang, C.; and Sun, J. 2017. Aligne-dreid: Surpassing human-level performance in person re- identification. arXiv preprint arXiv:1711.08184.

Zhong, Z.; Zheng, L.; Cao, D.; and Li, S. 2017. Re- ranking person re-identification with k-reciprocal encod- ing. 2017 IEEE Conference on Computer Vision and Pat- tern Recognition (CVPR) 3652-3661.

mentioned in \cite{Chen17a}:

S. Yu, Y. Yang, and A. Hauptmann, “Harry Potters Marauders Map: Localizing and tracking multiple persons-of-interest by nonnegative dis- cretization,” in Proc. IEEE Conf. Comput. Vis. Pattern Recognit. (CVPR), Jun. 2013, pp. 3714-3720.

mentioned in \cite{Chen17b}:

X. Chen, K. Huang, and T. Tan, “Object tracking across non-overlapping views by learning inter-camera transfer models,” Pattern Recognit., vol. 47, no. 3, pp. 1126-1137, 2014.

E. Reinhard, M. Adhikhmin, B. Gooch, and P. Shirley, “Color transfer between images,” IEEE Comput. Graph. Appl., vol. 21, no. 5, pp. 34-41, Sep./Oct. 2001.

M. Moussaiid, N. Perozo, S. Garnier, D. Helbing, and G. Theraulaz, The walking behaviour of pedestrian social groups and its impact on crowd dynamics

W. Ge, R. T. Collins, and R. B. Ruback, “Vision-based analysis of small groups in pedestrian crowds,” IEEE Trans. Pattern Anal. Mach. Intell., vol. 34, no. 5, pp. 1003-1016, May 2012.

D. Helbing and P. Molnar, Social force model for pedestrian dynamics, Phys. Rev. E, vol. 51, pp. 4282-4286, May 1995.

\subsection{Arising Questions}

Online Tracking?

Hungarian algorithm?

Multi Object vs Multi Target (definitions)

Attention mechanisms

Detection Frameworks:
\begin{itemize}
    \item YOLO
    \item Faster R-CNN
    \item R-CNN
\end{itemize}

Tracking Frameworks:
\begin{itemize}
    \item OpenCV
    \item DeepSORT
    \item SORT
    \item MOTSA
\end{itemize}