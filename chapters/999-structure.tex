\chapter{Structure}

\section{Citations}

\subsection{General}

\cite{Amosa23}: Current Trends in MCMOT. State of the Art. A lot of basic and advanced knowledge. Good for introduction. Analyzes 30 MCT algorithms.

\cite{Tian19}: General description of multi-camera tracking. State of the Art, Markov Process, graph partition theory, tracking by joint constraints.

\subsection{Real-time}

\cite{Wang21}: Indoor scene, multiple top-view \textbf{fisheye} cameras. Possible to cover large space, less occlusion among objects. People detection and tracking. Calibrate cameras, real time (FPS of about 10) without GPU support.

\cite{Chen22}: Real-time distributed MCMOT system. City-scale scenario. Keeping communication and computing costs of each device low. Installs smart stations on the roadside and connects them to maintain communication. Decentralized Tracking. Kalman filter and hungarian algorithm. YoloX and DeepSORT.

\subsection{VOT}

\cite{Kristan21}: VOT21 Challenge Results. Considers single-camera, single-target, model-free tracking. VOT-RT2021 focuses on real-time RGB tracking. Requires predicting bounding boxes. Top two trackers: TrasT\_M and STARK\_RT.

\cite{Kristan22}: VOT22 Challenge Results. Considers single.camera, single-target. VOT-RT2022 focuses on real-time RGB tracking, VOT-RTs by segmentation, VOT-RTb by bounding boxes. Goes beyond previous challenges (updating datasets). Real-time tracking at 20fps. Top trackers: MS\_AOT and OSTrackSTB.

\cite{Kristan23} VOTS23 Challenge Results. First year considering multiple-target tracking challenge. Explores short- and long-term at once. Only one challenge for all. Does not distinguish between these scenarios. Success is measured in IoU, tracking Quality \\mathbf{Q}, Accuracy, Robustness, NRE, DRE, ADQ. Dataset with challenging situations, wide range and diverse set of objects, object which are a part of other objects.  Also longer videos. 77 trackers submitted, 47 valid. Most trackers applied uniform dynamic model, utilized transformers, general segmentation network SAM. Top tracker: DMAOT built upon VOT22 winner AOT. Best segmentation-based trackers outperformed all bound.box trackers.


\subsection{Dynamic Cameras}
\cite{Nguyen22b}: Single-Stage Global Association Approach. Dynamic MCMOT (moving cameras in vehicle). Solves fragment-tracking issues. Not relevant for static MCMOT.

\subsection{Person Tracking}

\cite{Huang23}: Non-overlapping cameras. Pedestrian Tracking. Fix ID-switching issues with long-term feature extraction. OC-SORT + feature extraction.

\cite{Komorowski22}: Soccer Players. Raw detection heat maps. Google Research Football Environment. Multi camera, multi targets. Cameras have fixed positions. Do not use bounding boxes, instead raw input with heat maps. Graph Neural Network. No visual cues, such as jersey numbers. Player movement trajectories and interaction between neighborhood players.

\subsection{Vehicle Tracking (AI City)}

\cite{Specker22}: Multi-camera vehicle tracking. No real-time tracking. Improve single-camera tracklets. 4th place in 2022 AI City Challenge. Track refinement module. Yolov5 pre-trained on COCO. Using GAN to generate synthetic data. Background filtering. Hierarchical clustering, zones, two rounds of clustering (tracklets separately each possible transition between cameras, akk tracks fro adjacent cameras).

\cite{Lui21}: Inspired \cite{Specker22}. First place in 2021 AI City Challenge. Yolov5 pre-trained on COCO. Most important: Introduces two step clustering (inter-zone, inter-camera clustering).

\cite{Specker21}: Fourth place in 2021 AI City Challenge (Track 3). Occlusion-aware tracking system. Inspired by Stadler.

\cite{Li22a}: Second place in 2022 AI City Challenge (Track 1). No new innovations made on first glance.

\cite{Hsu22}: Graph Auto-Encoder and Self-Supervised Camera Link Model. First implementation of GAE in MTMCT. Very interesting paper. Network topology is learned automatically.

\subsection{Re-ID}

\cite{Li22b}: Proposes Mutual Information Temporal Weight Aggregated Person Re-ID Model (MI-TWA). Person re-identification. New algorithm. Not so interesting.

\subsection{Misc}

\cite{Zhang15a}: Tracking framework for multiple interacting targets both overlapping and non-overlapping cameras, raw target trajectory with group state. SVMS, homography-based voting schema, networkflow problem, K-shortest paths algorithm.

\cite{Choi16}: Non-overlapping multiple cameras tracking based on similarity function. Data association method. Similarity based on color appearance and camera topology. Use superpixels for extracting color features generated by Simple Linear Iterative Clustering K-means camera topology learning.

\cite{Nguyen22a}: Mathematical multi-camera tracking approach. Pre-clustering obtained from 3D geometry projections.

\cite{Cheng23}: Utilizes information regarding spatial and temporal consistency. Reconfigurable graph model. Two step approach: Associate all objects across cameras spatially then reconfig into a temporal graph model. Matching object across different views.

\section{Approaches}

Single vs Multi Camera Tracking

Static vs Dynamic MCMOT

Single-Stage vs Multi-Stage Tracking

Graph Neural Networks, Self-Attention, Transformers

Hierarchical Clustering

Tracking by detections vs Tracking by

Challenges: Occlusion, perspective changes, changes in lighting, changes in appearances, unknown number of targets in the whole network, unknown number of cameras in which a certain target appears.

Common Pipeline:
\begin{itemize}
    \item Detection
    \item Feature Extraction
    \item Single Camera Tracking
    \item Cross Camera Association
    \item Multi Camera Tracking
\end{itemize}
\section{Die Beschics}

Single Object Detection (SOD)

Multi Object Detection (MOD)

Object Re-Identification (ReID)

Single Camera Tracking (SCT)

Multi Camera Tracking (MCT)

Camera Link Model (CLM)

Trajectories and Tracklets

Fisheye vs Normal Cameras

Different cameras have different technical characteristics.

Appearance features vs Motion features

\section{Composition}

\begin{itemize}
    \item Introduction
    \item Motivation
    \item Technical Background
    \item Problem Statement
    \item State of the Art
    \item Approaches
    \item Challenges
    \item Papers
    \item Further Research
    \item Conclusion
\end{itemize}


\section{Research}

mentioned in \cite{Specker21}:

D. Stadler and J. Beyerer. Improving multiple pedestrian tracking by track management and occlusion handling. In IEEE Conf. Comput. Vis. Pattern Recog., 2021.